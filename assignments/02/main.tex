\documentclass[11pt, reqno]{amsart}

\input{~/latex-common/macros.tex}
\usepackage[backend=bibtex,style=science]{biblatex}
% \bibliography{main.bib}
\pgfplotsset{compat=1.18}

\pagestyle{fancy}                         % fancy (allow headers, footers)
\fancyhf{}                                % clear all header/footer settings.
\cfoot{\thepage}                          % set page-numbers in footer.
% \lhead{\textit{\textbf{ Amittai, S}}}   % set name in header, left.
% \rhead{\textsc{Math 71: Algebra}}       % set class name in header, right.
\renewcommand{\headrulewidth}{0pt}
\renewcommand{\footrulewidth}{0pt}


\renewcommand{\theenumi}{\alph{enumi}}

\begin{document}

\newdate{due-date}{31}{03}{2023}

\title{CS-49: Game Theory\\ Amittai Siavava \\ \displaydate{due-date}}
\author{Amittai Siavava}
% \date{\today}


\setlength{\headheight}{13.0pt}
\setlength{\footskip}{15.0pt}

\maketitle

\begin{problem}[2]
  You are offered the chance to collect $\$x$ tax-free if the toss of a fair coin
  comes up ``heads''; if it comes up tails you get nothing.
  Or, instead, you can have $\$y$ tax-free outright.\\
  Consider the six cases $x \in \set{10, 100, 1000, 10000, 100000, 1000000 }$.\\
  In each case, what value of $y$ would make it the toughest possible decision for you?

  \begin{table}[H]
    \centering
    \begin{tabular}{r|r|r|l}
      $x$ & $y$ & $\%$ & Reason \\
    \midrule
      $10$ & $100$ & $1000$ & Low winnings, play for fun (unless offered more money outright). \\
      $100$ & $100$ & $100$ & Payoff is still low. Play for fun unless offered more money outright. \\
      $1,000$ & $500$ & $50$ & Higher potential win justifies taking bigger risk. \\
      $10,000$ & $1,000$ & $10$ & Higher potential win justifies taking bigger risk. \\
      $100,000$ & $50,000$ & $50$ & Higher potential win ustifies taking bigger risk. \\
      $1,000,000$ & $300,000$ & $30$ & Higher potential win justifies taking bigger risk. \\
    \toprule
    \end{tabular}
    \caption{Potential Payoff vs. Outright Money.}
  \end{table}
\end{problem}
\end{document}
