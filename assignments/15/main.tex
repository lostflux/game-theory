\documentclass[11pt, reqno]{amsart}

\input{~/latex-common/macros.tex}
\usepackage[backend=bibtex,style=science]{biblatex}
% \bibliography{main.bib}
\pgfplotsset{compat=1.18}

\pagestyle{fancy}                         % fancy (allow headers, footers)
\fancyhf{}                                % clear all header/footer settings.
\cfoot{\thepage}                          % set page-numbers in footer.
% \lhead{\textit{\textbf{ Amittai, S}}}   % set name in header, left.
% \rhead{\textsc{Math 71: Algebra}}       % set class name in header, right.
\renewcommand{\headrulewidth}{0pt}
\renewcommand{\footrulewidth}{0pt}


\renewcommand{\theenumi}{\alph{enumi}}

\begin{document}

\newdate{due-date}{03}{05}{2023}

\title{CS-49: Game Theory\\ Amittai Siavava \\ \displaydate{due-date}}
\author{Amittai Siavava}
% \date{\today}


\setlength{\headheight}{13.0pt}
\setlength{\footskip}{15.0pt}

\maketitle

\bigskip

\def \cram { \textsc{cram} }
\def \dom { \textsc{domineering} }
\def \sub { \textsc{subtraction} }
\def \weighted { \textsc{weighted odds and evens}}
\def \nim { \textsc{nim} }
\def \P { \mathbf{P}}
\def \N { \mathbf{N}}
\def \cc { \mathbf{cc} }

\newpage
\begin{problem}[15]
  The convex closure $\cc(U)$ of a set $U \subseteq E_d$
  is the intersection of all convex sets containing $U$. Show that $\cc(U)$
  can also be described as the set of all points $c_1 p_1 + c_2 p_2 + \ldots + c_n p_n$
  where $n$ is any positive integer, each $p_i$ is any point (i.e., d-tuple) of $U$,
  and the $c_i$'s are arbitrary nonnegative reals that sum to $1$.

  \step
  A convex set is a set such that for any two points in the set, every point on the line
  segment connecting them is also in the set.
  While $U$ need not be convex, $\cc(U)$ is the smallest convex set containing $U$.
  This means that $\cc(U)$ is the set of all points in $U$ and all points on the lines
  between any two points in $U$. Thus, every point in $\cc(U)$ is either:
  \begin{enumroman}
    \item A member of $U$. In this case, we can write the point $p$ as $1 \cdot p$.
    \item The point is not a member of $U$, but is on the line segment between some
      two points in $U$.
      Let $p_1$ and $p_2$ be two points in $U$, and $p_3$ be any point on the line segment
      between $p_1$ and $p_2$, then we can write $p_3 = p_1 + \lambda (p_2 - p_1)$ for some
      $\lambda \in [0,1]$.
      Refactoring, we see that $p_3 = (1 - \lambda) p_1 + \lambda p_2$. Most importantly,
      $c_1 = 1 - \lambda$ and $c_2 = \lambda$ are nonnegative and sum to $1$.
      Suppose $p_4 \in U$ is any other point, then every point on the line segment between
      $p_3$ and $p_4$ is also contained in $\cc(U)$. Using the same process as we did for $p_3$,
      if $p_5$ is any point on the line segment between $p_3$ and $p_4$, then we can write
      $p_5 = (1 - \mu) p_3 + \mu p_4$ for some $\mu \in [0,1]$. Since
      $p_3 = (1 - \lambda) p_1 + \lambda p_2$, we can write
      \[ p_5 \quad = 
        \quad (1 - \mu) ((1 - \lambda) p_1 + \lambda p_2) + \mu p_4 \quad =
        \quad (1 - \mu)(1 - \lambda) p_1 + (1 - \mu)\lambda p_2 + \mu p_5.
      \]
      % \step
      Since \crim{$(1 - \lambda) + \lambda = 1$},
      \[ \zaff{(1 - \mu)(1 - \lambda) + (1 - \mu)\lambda} = \green{(1 - \mu)(\crim{(1 - \lambda) + \lambda})} = 1 - \mu. \]
      On the other hand, $(1 - \mu) + \mu = 1$, so the constants in the equation for $p_5$ sum to $1$.
      Using the same logic, we can extend the equation to all points between $p_5$ and any other point
      in $U$, since any such point on the line segment must be contained in $\cc(U)$.
      Therefore, every point in $\cc(U)$ can be written as such a combination of points in $U$
      with nonnegative coefficients that sum to $1$.
  \end{enumroman}
\end{problem}
\end{document}
