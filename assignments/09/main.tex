\documentclass[11pt, reqno]{amsart}

\input{~/latex-common/macros.tex}
\usepackage[backend=bibtex,style=science]{biblatex}
% \bibliography{main.bib}
\pgfplotsset{compat=1.18}

\pagestyle{fancy}                         % fancy (allow headers, footers)
\fancyhf{}                                % clear all header/footer settings.
\cfoot{\thepage}                          % set page-numbers in footer.
% \lhead{\textit{\textbf{ Amittai, S}}}   % set name in header, left.
% \rhead{\textsc{Math 71: Algebra}}       % set class name in header, right.
\renewcommand{\headrulewidth}{0pt}
\renewcommand{\footrulewidth}{0pt}


\renewcommand{\theenumi}{\alph{enumi}}

\begin{document}

\newdate{due-date}{19}{04}{2023}

\title{CS-49: Game Theory\\ Amittai Siavava \\ \displaydate{due-date}}
\author{Amittai Siavava}
% \date{\today}


\setlength{\headheight}{13.0pt}
\setlength{\footskip}{15.0pt}

\maketitle

\bigskip

\def \cram { \textsc{cram} }
\def \dom { \textsc{domineering} }
\def \sub { \textsc{subtraction} }
\def \weighted { \textsc{weighted odds and evens}}

\newpage
\begin{problem}[9]
  Let $x$ be the value of some particular impartial game.
  Then, as we saw in class, $x + x = 0$, since, by symmetrizing,
  the sum of two copies of this game is in the class $P$ whose value is $0$
  Since addition of games is commutative (and associative),
  the set of all possible values of impartial games should be an abelian group
  in which $x + x = 0$ for all elements $x$.
  
  Find an example of such a group, preferably one of countably infinite size.
  
  \medskip
  The simplest group $G$ in which every non-identity element $x$ has order $2$
  is the commutative group of order $2$, \[ C_2 = \parens{\set{0,1};; +, 0} \isom \Z/2\Z.\]

  We can generate higher-order groups by taking repeated direct products
  of $C_2$ with itself.
  For instance, \[ C_2 \times C_2 = \parens{\set{(0,0),(0,1),(1,0),(1,1)}; +, (0,0)}. \]
  Through a little inspection, we can easily see that this group is isomorphic
  to the \emph{Klein four} group, $V_4$, since each non-identity element has order
  $2$ and $(1,1) = (0,1) + (1,0)$.
  To get a countably infinite group, consider the group generated by taking
  an infinite number of direct products of $C_2$ with itself.
  Such groups have similar structures to the corresponding vector spaces over $C_2$ ---
  for example, $V_4$ is isomorphic to the vector space of dimension $2$ over $C_2$:
  \begin{align*}
    \bV = \set{
      \begin{bmatrix} 0 \\ 0 \end{bmatrix},
      \begin{bmatrix} 0 \\ 1 \end{bmatrix},
      \begin{bmatrix} 1 \\ 0 \end{bmatrix},
      \begin{bmatrix} 1 \\ 1 \end{bmatrix} },
      \quad
      \mathbf{basis}(\bV) = \set{
        \begin{bmatrix} 0 \\ 1 \end{bmatrix},
        \begin{bmatrix} 1 \\ 0 \end{bmatrix}
      }
  \end{align*}

  Similarly, an infinite-dimensional group in which every element
  has order $2$ is isomorphic to an infinite-dimensional vector space
  over $\Z/2\Z$. Since the basis (and therefore the vector space itself)
  can be ordered in a countable way, the corresponding group elements
  can also be ordered in a countable way.
\end{problem}
\end{document}
