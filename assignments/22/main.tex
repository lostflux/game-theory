\documentclass[11pt, reqno]{amsart}

\input{~/latex-common/macros.tex}
\usepackage[backend=bibtex,style=science]{biblatex}
% \bibliography{main.bib}
\pgfplotsset{compat=1.18}

\pagestyle{fancy}                         % fancy (allow headers, footers)
\fancyhf{}                                % clear all header/footer settings.
\cfoot{\thepage}                          % set page-numbers in footer.
% \lhead{\textit{\textbf{ Amittai, S}}}   % set name in header, left.
% \rhead{\textsc{Math 71: Algebra}}       % set class name in header, right.
\renewcommand{\headrulewidth}{0pt}
\renewcommand{\footrulewidth}{0pt}


\renewcommand{\theenumi}{\alph{enumi}}

\begin{document}

\newdate{due-date}{22}{05}{2023}

\title{CS-49: Game Theory\\ Amittai Siavava \\ \displaydate{due-date}}
\author{Amittai Siavava}
% \date{\today}


\setlength{\headheight}{13.0pt}
\setlength{\footskip}{15.0pt}

\maketitle

\bigskip

\def \cram { \textsc{cram} }
\def \dom { \textsc{domineering} }
\def \sub { \textsc{subtraction} }
\def \weighted { \textsc{weighted odds and evens}}
\def \nim { \textsc{nim} }
\def \P { \mathbf{P}}
\def \N { \mathbf{N}}
\def \cc { \mathbf{cc} }
\def \hackenbush { \textsc{LR hackenbush} }
\def \clobber { \textsc{clobber} }

\newpage
  \begin{problem}[22]
    Let $S$ be a board of size $2n$ for a selection game $G$ in which,
    prior to each move, the next card is drawn from a well-shuffled deck
    of $n$ blue cards and $n$ red cards.
    The next move is then taken by Louise if the card is blue, Richard if red.
    The payoff to Louise is $f(S_1)$, where $f$ is a function from subsets of $S$
    of size $n$ to the real numbers, and the payoff to Richard is $-f(S_1)$,
    where $S_1$ is the set Louise ends up with.
    Show that the (expected) value to Louise of this game is the expected value of $f(R)$,
    where $R$ is a uniformly random subset of $S$ of size $n$.

    \step
    In each turn, the payoff is determined by the set $S$ and not the who makes the move.
    The expected value to Louise is the averaged value of all such sets of size $n$.
    The number of such possible sets of size $n$ is $\binom{2n}{n}$.
    Therefore, the probability of any one subset of $2^S$ of size $n$ being selected
    is $1/\binom{2n}{n}$. The expected value of the game to Louise is
    \[ \mathbf{value}(G) = \sum_{i = 1}^{\binom{2n}{n}} \frac{1}{\binom{2n}{n}} f(S_i) \]

    \step
    Now, consider if a uniformly random $R \subset S$ of size $n$ is selected.
    There are $2n$ possible candidates for the first element of the subset,
    $2n-1$ for the second, and so on up to the $n$'th.
    In total, there are $\binom{2n}{n}$ possible configurations of subsets of size $n$.
    The probability of $R$ being any one specific subset is $1/\binom{2n}{n}$.
    Therefore, the expected value of $R$ is:
    \[ f(R) = \sum_{i = 1}^{\binom{2n}{n}} \frac{1}{\binom{2n}{n}} f(S_i) \]

    \step
    Therefore, \[ \mathbf{value}(G) = f(R). \]
    
    \vfill
    \step
    \textbf{References}
    \begin{enumroman}
      \item For random turn games, see \url{https://arxiv.org/pdf/math/0508580.pdf}.
      \item For Knuth's Surreal Numbers, see \\ \url{
        https://people.math.harvard.edu/~knill/teaching/
        mathe320_2015_fall/blog15/surreal1.pdf
      }.
    \end{enumroman}
  \end{problem}
\end{document}
