\documentclass[11pt, reqno]{amsart}

\input{~/latex-common/macros.tex}
\usepackage[backend=bibtex,style=science]{biblatex}
% \bibliography{main.bib}
\pgfplotsset{compat=1.18}

\pagestyle{fancy}                         % fancy (allow headers, footers)
\fancyhf{}                                % clear all header/footer settings.
\cfoot{\thepage}                          % set page-numbers in footer.
% \lhead{\textit{\textbf{ Amittai, S}}}   % set name in header, left.
% \rhead{\textsc{Math 71: Algebra}}       % set class name in header, right.
\renewcommand{\headrulewidth}{0pt}
\renewcommand{\footrulewidth}{0pt}


\renewcommand{\theenumi}{\alph{enumi}}

\begin{document}

\newdate{due-date}{31}{05}{2023}

\title{CS-49: Game Theory\\ Amittai Siavava \\ \displaydate{due-date}}
\author{Amittai Siavava}
% \date{\today}


\setlength{\headheight}{13.0pt}
\setlength{\footskip}{15.0pt}

\maketitle

\bigskip

\def \cram { \textsc{cram} }
\def \dom { \textsc{domineering} }
\def \sub { \textsc{subtraction} }
\def \weighted { \textsc{weighted odds and evens}}
\def \nim { \textsc{nim} }
\def \P { \mathbf{P}}
\def \N { \mathbf{N}}
\def \cc { \mathbf{cc} }
\def \hackenbush { \textsc{LR hackenbush} }
\def \clobber { \textsc{clobber} }

% \newpage
  \begin{problem}[25]
    $n$ people are involved in a \emph{pivotal mechanism} (see top half of \textbf{page 299} in Tadelis)
    to decide where to locate a prison. Person $i$ is just willing to pay $d_i$ to avoid having the prison in his town,
    where $0 < d_1 < d_2 < \ldots < d_n$.
    The prison has to go somewhere.
    Where does it go, and what does each player earn or pay?

    \step
    If we use the amount each town is willing to pay to keep the prison out as a measure of overall `discomfort'
    associated with locating the prison in that town, then the prison should go to the town with the least discomfort,
    person $1$'s town.
    Since the prison does not end up in any of towns $2, 3, \ldots, n$, the representative for each of those towns
    initially pays the amount $d_2$, $d_3$, $\ldots$, $d_n$. The amount is then distributed equally among the $n$ people.
    That is, each person receives \[ \frac{1}{n} \sum_{2}^{n} d_i. \]
    Thus, the representative for town $k$ pays
    \[
      \mathbf{pay}(k) = \begin{cases}
        - \frac{1}{n} \sum_{2}^{n} d_i & \text{if $k = 1$} \\
        d_k - \frac{1}{n} \sum_{2}^{n} d_i & \text{if $k \neq 1$}
      \end{cases},
    \]
    where a negative amount means the person \emph{receives} that amount, and a positive amount means 
    the person \emph{pays} that amount.
  \end{problem}
\end{document}
