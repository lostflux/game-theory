\documentclass[11pt, reqno]{amsart}

\input{~/latex-common/macros.tex}
\usepackage[backend=bibtex,style=science]{biblatex}
% \bibliography{main.bib}
\pgfplotsset{compat=1.18}

\pagestyle{fancy}                         % fancy (allow headers, footers)
\fancyhf{}                                % clear all header/footer settings.
\cfoot{\thepage}                          % set page-numbers in footer.
% \lhead{\textit{\textbf{ Amittai, S}}}   % set name in header, left.
% \rhead{\textsc{Math 71: Algebra}}       % set class name in header, right.
\renewcommand{\headrulewidth}{0pt}
\renewcommand{\footrulewidth}{0pt}


\renewcommand{\theenumi}{\alph{enumi}}

\begin{document}

\newdate{due-date}{12}{04}{2023}

\title{CS-49: Game Theory\\ Amittai Siavava \\ \displaydate{due-date}}
\author{Amittai Siavava}
% \date{\today}


\setlength{\headheight}{13.0pt}
\setlength{\footskip}{15.0pt}

\maketitle

\bigskip

\def \cram { \textsc{cram} }
\def \dom { \textsc{domineering} }
\def \sub { \textsc{subtraction} }
\def \weighted { \textsc{weighted odds and evens}}

\newpage
\begin{problem}[7]
  In \weighted, Alice and Bob simultaneously put out one or two fingers.
  If they put out different numbers of fingers,
  Alice wins from Bob a number of dollars equal to the total number of fingers
  put out (namely, in this case, 3).
  If they put out the same number of fingers, Alice pays Bob $\$2$ or $\$4$
  according to the total number of fingers played.
  
  \begin{enumalph}
    \item Find an equilibrium pair of randomized strategies. 
      \step
      \begin{align*}
        % \mathbb{P} &=
        % \begin{bmatrix}
        %   (-2, +2) & (+3, -3) \\ (+3, -3) & (-4, +4)
        % \end{bmatrix}\\ \\
        A &= \begin{bmatrix}
          -2 & +3 \\ +3 & -4
        \end{bmatrix}\\ \\
        B &= \begin{bmatrix}
          +2 & -3 \\ -3 & +4
        \end{bmatrix}
      \end{align*}

      \step
      Alice has two options:
      \begin{enumroman}
        \item Put out one finger. Half the time, she loses $\$2$,
          and half the time she wins $\$3$, for an expected value
          of $\$(0.5 \cdot (-2) + 0.5 \cdot (+3)) = \$+0.5$ per turn.
        \item Put out two fingers.
          Half the time, she loses $\$4$,
          and half the time she wins $\$3$ for an expected value of
          $\$(0.5 \cdot (-4) + 0.5 \cdot (+3)) = \$-0.5$ per turn.
      \end{enumroman}
      Alice is better off putting out one finger.

      \step
      Bob has two options:
      \begin{enumroman}
        \item Put out one finger. Half the time, he wins $\$2$,
          and half the time he loses $\$3$ for an expected value
          of $\$(0.5 \cdot (+2) + 0.5 \cdot (-3)) = \$-0.5$
          per turn.
        \item Put out two fingers.
          Half the time, he wins $\$4$,
          and half the time he loses $\$3$ for an expected value of
          $\$(0.5 \cdot (+4) + 0.5 \cdot (-3)) = \$+0.5$ per turn.
      \end{enumroman}
      Bob is better off putting out two fingers.

      \medskip
    \item What is the expected outcome if these strategies are employed?
    
      \step
      In the equilibrium pair of strategies, Alice puts out one finger
      and Bob puts out two fingers.
      Since the fingers do not match, Alice wins $\$+3$ per turn.
  \end{enumalph}


  
\end{problem}
\end{document}
