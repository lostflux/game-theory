\documentclass[11pt, reqno]{amsart}

\input{~/latex-common/macros.tex}
\usepackage[backend=bibtex,style=science]{biblatex}
% \bibliography{main.bib}
\pgfplotsset{compat=1.18}

\pagestyle{fancy}                         % fancy (allow headers, footers)
\fancyhf{}                                % clear all header/footer settings.
\cfoot{\thepage}                          % set page-numbers in footer.
% \lhead{\textit{\textbf{ Amittai, S}}}   % set name in header, left.
% \rhead{\textsc{Math 71: Algebra}}       % set class name in header, right.
\renewcommand{\headrulewidth}{0pt}
\renewcommand{\footrulewidth}{0pt}


\renewcommand{\theenumi}{\alph{enumi}}

\begin{document}

\newdate{due-date}{10}{04}{2023}

\title{CS-49: Game Theory\\ Amittai Siavava \\ \displaydate{due-date}}
\author{Amittai Siavava}
% \date{\today}


\setlength{\headheight}{13.0pt}
\setlength{\footskip}{15.0pt}

\maketitle

\bigskip

\def \cram { \textsc{cram} }
\def \dom { \textsc{domineering} }
\def \sub { \textsc{subtraction} }

% \newpage
\begin{problem}[6]
  Determine (with proof!) the $\mathbf{P}$ and $\mathbf{N}$ positions for \sub,
  where the set of subtractibles is $\set{3,4,5}$.

  \step
  Recall that;
  \begin{enumarabic}
    \item A position is in class $\mathbf{P}$ if \emph{all} possible moves end in positions
      in class $\mathbf{N}$.
    \item A position is in class $\mathbf{N}$ if there exists a move that ends in a
      position in class $\mathbf{P}$.\\
      \emph{Note that only a single move is needed!}
  \end{enumarabic}

  By writing out the possible values for the first $23$ moves (see table on next page),
  we see this general pattern emerge:

  \[\mathbf{P} = \set{ 0, 1, 2} \cup \set{8, 9, 10} \cup \set{16, 17, 18} \cup \ldots \]
  \[\mathbf{N} = \set{3, 4, 5, 6, 7} \cup \set{11, 12, 13, 14, 15} \cup \set{19, 20, 21, 22, 23} \cup \ldots \]

  \step
  In general, a given position $n$ is in the class $\mathbf{P}$ iff $n \pmod{8} < 3$,
  and a given position $n$ is in the class $\mathbf{N}$ iff $n \pmod{8} \ge 3$.

  \begin{table}[h!]
    \centering
    \begin{tabular}{||r|r|r|r|r||}
    \bottomrule
      Current & \multicolumn{3}{|c|}{Next} & P or N? \\
    \midrule
      $0$ & $\green{-3}$ & $\green{-4}$ & $\green{-5}$ & $\mathbf{P}$ \\
      $1$ & $\green{-2}$ & $\green{-3}$ & $\green{-4}$ & $\mathbf{P}$ \\
      $2$ & $\green{-1}$ & $\green{-2}$ & $\green{-3}$ & $\mathbf{P}$ \\
      \midrule
      $3$ & $\crim{0}$ & $\green{-1}$ & $\green{-2}$ & $\mathbf{N}$ \\
      $4$ & $\crim{1}$ & $\crim{0}$ & $\green{-1}$ & $\mathbf{N}$ \\
      $5$ & $\crim{2}$ & $\crim{1}$ & $\crim{0}$ & $\mathbf{N}$ \\
      $6$ & $\green{3}$ & $\crim{2}$ & $\crim{1}$ & $\mathbf{N}$ \\
      $7$ & $\green{4}$ & $\green{3}$ & $\crim{2}$ & $\mathbf{N}$ \\
      \midrule
      $8$ & $\green{5}$ & $\green{4}$ & $\green{3}$ & $\mathbf{P}$ \\
      $9$ & $\green{6}$ & $\green{5}$ & $\green{4}$ & $\mathbf{P}$ \\
      $10$ & $\green{7}$ & $\green{6}$ & $\green{5}$ & $\mathbf{P}$ \\
      \midrule
      $11$ & $\crim{8}$ & $\green{7}$ & $\green{6}$ & $\mathbf{N}$ \\
      $12$ & $\crim{9}$ & $\crim{8}$ & $\green{7}$ & $\mathbf{N}$ \\
      $13$ & $\crim{10}$ & $\crim{9}$ & $\crim{8}$ & $\mathbf{N}$ \\
      $14$ & $\green{11}$ & $\crim{10}$ & $\crim{9}$ & $\mathbf{N}$ \\
      $15$ & $\green{12}$ & $\green{11}$ & $\crim{10}$ & $\mathbf{N}$ \\
      \midrule
      $16$ & $\green{13}$ & $\green{12}$ & $\green{11}$ & $\mathbf{P}$ \\
      $17$ & $\green{14}$ & $\green{13}$ & $\green{12}$ & $\mathbf{P}$ \\
      $18$ & $\green{15}$ & $\green{14}$ & $\green{13}$ & $\mathbf{P}$ \\
      \midrule
      $19$ & $\crim{16}$ & $\green{15}$ & $\green{14}$ & $\mathbf{N}$ \\
      $20$ & $\crim{17}$ & $\crim{16}$ & $\green{15}$ & $\mathbf{N}$ \\
      $21$ & $\crim{18}$ & $\crim{17}$ & $\crim{16}$ & $\mathbf{N}$ \\
      $22$ & $\green{19}$ & $\crim{18}$ & $\crim{17}$ & $\mathbf{N}$ \\
      $23$ & $\green{20}$ & $\green{19}$ & $\crim{18}$ & $\mathbf{N}$ \\
      \midrule
      $\vdots$ & $\vdots$ & $\vdots$ & $\vdots$ & $\vdots$ \\
    \end{tabular}
    \caption{Potential Payoff vs. Outright Money.}
  \end{table}
\end{problem}
\end{document}
