\documentclass[11pt, reqno]{amsart}

\input{~/latex-common/macros.tex}
\usepackage[backend=bibtex,style=science]{biblatex}
% \bibliography{main.bib}
\pgfplotsset{compat=1.18}

\pagestyle{fancy}                         % fancy (allow headers, footers)
\fancyhf{}                                % clear all header/footer settings.
\cfoot{\thepage}                          % set page-numbers in footer.
% \lhead{\textit{\textbf{ Amittai, S}}}   % set name in header, left.
% \rhead{\textsc{Math 71: Algebra}}       % set class name in header, right.
\renewcommand{\headrulewidth}{0pt}
\renewcommand{\footrulewidth}{0pt}


\renewcommand{\theenumi}{\alph{enumi}}

\begin{document}

\newdate{due-date}{28}{04}{2023}

\title{CS-49: Game Theory\\ Amittai Siavava \\ \displaydate{due-date}}
\author{Amittai Siavava}
% \date{\today}


\setlength{\headheight}{13.0pt}
\setlength{\footskip}{15.0pt}

\maketitle

\bigskip

\def \cram { \textsc{cram} }
\def \dom { \textsc{domineering} }
\def \sub { \textsc{subtraction} }
\def \weighted { \textsc{weighted odds and evens}}
\def \nim { \textsc{nim} }
\def \P { \mathbf{P}}
\def \N { \mathbf{N}}

\newpage
\begin{problem}[13]
  State and prove the $n$-dimensional Sperner's Lemma, by induction on $n$.

  \begin{lemma}
    If an $n$-dimensional simplex is colored with $n+1$ colors,
    such that each of its $n-1$ facets is colored with $n$ colors
    having the $n$ endpoints of the facet differently colored,
    then the number of $n$-dimensional simplices with all $n+1$ colors
    is odd.
  \end{lemma}

  \step
  The $2$-dimensional Sperner's Lemma states that if a triangle with vertices colored
  $1$,$2$,$3$ is "triangulated" into smaller triangles, and vertices are colored
  $1$, $2$ or $3$ with the stipulation that outside vertices
  (that is, vertices on an edge of the big triangle) can only get a color
  of one of the big edge's endpoints, then an odd number of cells (small triangles) get all the colors.

  \step
  As we showed in class, the $1$-dimensional Sperner's Lemma is true since,
  when applied to a line segment with the condition that the two endpoints
  of the line segment are not colored the same, then there must be an odd
  number of rainbow segments.

  \step
  In the $2$-dimensional segment, we have $3$ colors and a triangle shape.
  Each edge of the triangle is a $1$-dimensional instance of Sperner's Lemma,
  since only the colors at the two endpoints are permitted on points along the edge,
  and the edges must once again be colored differently.
  By induction, assume that Sperner's Lemma is valid for the $1$-dimensional case,
  then each edge of the $2$-dimensional triangle must have an odd number of rainbow segments.
  If we draw a path from one rainbow segment inwards,
  with the condition that all segments the path crosses are colored with those same two colors,
  then the path must either be halted by a rainbow triangle,
  or it must exit the triangle on the same edge (since all the other edges
  strictly do not have that pair of coloring on a segment).
  Given that the segment has an odd number of rainbow segments,
  and all segments that exit will result in a pairing of segments, the number that
  are stopped must be odd, so the number of rainbow triangles must be odd.

  \step
  In the $n$-dimensional case, the $n$-dimensional simplex is composed of
  $n+1$ facets, each of dimension $n-1$. It also has $n+1$ colors.
  Supposing that the $n-1$-dimensional Sperner's Lemma is true,
  then each facet must have an odd number of rainbow simplices.
  If we draw a path from one rainbow simplex inwards,
  with the condition that all simplices the path crosses are colored with those same $n-1$ colors,
  then the path must either be halted by a rainbow simplex,
  or it must exit the simplex on the same facet (since all the other facets
  strictly have some other color not present on this facet).
  The facet has odd rainbow simplices, and those paths that enter and exit the facet
  pair up rainbow simplices, so the number of stopped paths must be odd, meaning the
  $n+1$-dimensional simplex must have also an odd number of rainbow simplices.

\end{problem}
\end{document}
