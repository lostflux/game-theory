\documentclass[11pt, reqno]{amsart}

\input{~/latex-common/macros.tex}
\usepackage[backend=bibtex,style=science]{biblatex}
% \bibliography{main.bib}
\pgfplotsset{compat=1.18}

\pagestyle{fancy}                         % fancy (allow headers, footers)
\fancyhf{}                                % clear all header/footer settings.
\cfoot{\thepage}                          % set page-numbers in footer.
% \lhead{\textit{\textbf{ Amittai, S}}}   % set name in header, left.
% \rhead{\textsc{Math 71: Algebra}}       % set class name in header, right.
\renewcommand{\headrulewidth}{0pt}
\renewcommand{\footrulewidth}{0pt}


\renewcommand{\theenumi}{\alph{enumi}}

\begin{document}

\newdate{due-date}{24}{04}{2023}

\title{CS-49: Game Theory\\ Amittai Siavava \\ \displaydate{due-date}}
\author{Amittai Siavava}
% \date{\today}


\setlength{\headheight}{13.0pt}
\setlength{\footskip}{15.0pt}

\maketitle

\bigskip

\def \cram { \textsc{cram} }
\def \dom { \textsc{domineering} }
\def \sub { \textsc{subtraction} }
\def \weighted { \textsc{weighted odds and evens}}
\def \nim { \textsc{nim} }
\def \P { \mathbf{P}}

\newpage
\begin{problem}[11]
  Replace each of the first five letters of your official Dartmouth email address
  by its position in the alphabet (a number between $1$ and $26$),
  and consider the resulting $5$-stack \nim position.
  Find all winning moves (if any) from that position.

  % double columns
  \begin{multicols}{2}
    \[
      \text{letters:} \texttt{amitt}
    \]

    Converting to binary:
    \begin{align*}
      \textsc{A} = 1 &=  00001 \\
      \textsc{M} = 13 &= 01101 \\
      \textsc{I} = 9 &=  01001 \\
      \textsc{T} = 20 &= 10100 \\
      \textsc{T} = 20 &= 10100 \\
    \end{align*}
    
    \begin{figure}[H]
      \begin{tabular}{c@{\,}c@{\,}c@{\,}c@{\,}c@{\,}|c}
        0 & 0 & 0 & 0 & 1 & A\\
        0 & 1 & 1 & 0 & 1 & M\\
        0 & 1 & 0 & 0 & 1 & I\\
        1 & 0 & 1 & 0 & 0 & T\\
        1 & 0 & 1 & 0 & 0 & T\\
        \midrule
        0 & 0 & 1 & 0 & 1  
      \end{tabular}
      \caption{Initial game value.}
    \end{figure}
    A winning move needs to change the game's value to $0$,
    meaning it must make position $1$ and $3$ have an even number of
    $1$s across the five stacks.
    With the limitation to deduction, only one possible move achieves that:
    \begin{enumarabic}
      \item Remove $5$ stones from stack $2$ (for letter \textsc{M}).
        This changes the value of the stack from 01101 (equal to 13, or \textsc{M})
        to 01000 (equal to 8, or \textsc{H}) with this new value;

        \begin{figure}[H]
          \begin{tabular}{c@{\,}c@{\,}c@{\,}c@{\,}c@{\,}|c}
            0 & 0 & 0 & 0 & 1 & A\\
            0 & 1 & 0 & 0 & 0 & H\\
            0 & 1 & 0 & 0 & 1 & I\\
            1 & 0 & 1 & 0 & 0 & T\\
            1 & 0 & 1 & 0 & 0 & T\\
            \midrule
            0 & 0 & 0 & 0 & 0  
          \end{tabular}
          \caption{Game value after move.}
        \end{figure}
        \step
        This move goes from \textsc{amitt} to \textsc{ahitt}.
        Since the new game state has value $0$, it is in class $P$
        and is therefore a winning position for the player who just moved.
    \end{enumarabic}
  \end{multicols}
\end{problem}
\end{document}
