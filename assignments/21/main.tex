\documentclass[11pt, reqno]{amsart}

\input{~/latex-common/macros.tex}
\usepackage[backend=bibtex,style=science]{biblatex}
% \bibliography{main.bib}
\pgfplotsset{compat=1.18}

\pagestyle{fancy}                         % fancy (allow headers, footers)
\fancyhf{}                                % clear all header/footer settings.
\cfoot{\thepage}                          % set page-numbers in footer.
% \lhead{\textit{\textbf{ Amittai, S}}}   % set name in header, left.
% \rhead{\textsc{Math 71: Algebra}}       % set class name in header, right.
\renewcommand{\headrulewidth}{0pt}
\renewcommand{\footrulewidth}{0pt}


\renewcommand{\theenumi}{\alph{enumi}}

\begin{document}

\newdate{due-date}{19}{05}{2023}

\title{CS-49: Game Theory\\ Amittai Siavava \\ \displaydate{due-date}}
\author{Amittai Siavava}
% \date{\today}


\setlength{\headheight}{13.0pt}
\setlength{\footskip}{15.0pt}

\maketitle

\bigskip

\def \cram { \textsc{cram} }
\def \dom { \textsc{domineering} }
\def \sub { \textsc{subtraction} }
\def \weighted { \textsc{weighted odds and evens}}
\def \nim { \textsc{nim} }
\def \P { \mathbf{P}}
\def \N { \mathbf{N}}
\def \cc { \mathbf{cc} }
\def \hackenbush { \textsc{LR hackenbush} }
\def \clobber { \textsc{clobber} }

\newpage
  \begin{problem}[21]
    Take any partizan game $G$ with blue and red pieces (e.g., \clobber)
    and create an equivalent game $G'$ by having both players play the blue pieces,
    but the pieces switch colors after each move.
    Then $G'$ is impartial, right?
    So how can it be equivalent to a partizan game, like the \clobber position
    with value ``up'' shown in class?

    \step
    $G'$ is not impartial.
    We may assume that $G'$ is impartial because moves at any given position
    start with the given active color moving first
    (for example, if the active color is blue, then moves are always made on blue,
    but after each move the blues become red and the reds become blue).
    However, since the colors always change after each move, the first player
    and the second player effectively play different pieces, therefore
    have different moves.
    The game is not truly impartial because $\P$'s moves are not immediately
    available to $\N$ or vice versa.

    \step
    However, recoloring helps simplify game analysis since, if we know the active color
    (based on which color moved first, and how many color changes have occurred),
    then we may only focus on the relevant moves for that color in the current turn,
    the relevant moves for the other color in the next turn, and so on.
  \end{problem}
\end{document}
