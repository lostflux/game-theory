\documentclass[11pt, reqno]{amsart}

\input{~/latex-common/macros.tex}
\usepackage[backend=bibtex,style=science]{biblatex}
% \bibliography{main.bib}
\pgfplotsset{compat=1.18}

\pagestyle{fancy}                         % fancy (allow headers, footers)
\fancyhf{}                                % clear all header/footer settings.
\cfoot{\thepage}                          % set page-numbers in footer.
% \lhead{\textit{\textbf{ Amittai, S}}}   % set name in header, left.
% \rhead{\textsc{Math 71: Algebra}}       % set class name in header, right.
\renewcommand{\headrulewidth}{0pt}
\renewcommand{\footrulewidth}{0pt}


\renewcommand{\theenumi}{\alph{enumi}}

\begin{document}

\newdate{due-date}{15}{05}{2023}

\title{CS-49: Game Theory\\ Amittai Siavava \\ \displaydate{due-date}}
\author{Amittai Siavava}
% \date{\today}


\setlength{\headheight}{13.0pt}
\setlength{\footskip}{15.0pt}

\maketitle

\bigskip

\def \cram { \textsc{cram} }
\def \dom { \textsc{domineering} }
\def \sub { \textsc{subtraction} }
\def \weighted { \textsc{weighted odds and evens}}
\def \nim { \textsc{nim} }
\def \P { \mathbf{P}}
\def \N { \mathbf{N}}
\def \cc { \mathbf{cc} }
\def \hackenbush { \textsc{LR hackenbush} }

\newpage
\begin{problem}[19]
  We proved in class that no position in \hackenbush
  (AKA \textsc{BLUE-RED hackenbush}) can be in the outcome class $\N$.
  The proof seems to work even if some edges are green,
  but something's wrong here, because there are $\N$ positions in general
  \hackenbush.
  
  \begin{enumarabic}
    \item Where does our classroom proof go wrong when green edges are present?
    
      \step
      Since green edges can be chopped by either player,
      if there's a single green edge with all the remaining
      edges (red or blue) held by the green edge then the position is in $\N$.
    \item Can you nonetheless extend our theorem to some situations when
      green edges are present?

      \step
      For the theorem to work, there must be an even number of green edges
      in the position.
  \end{enumarabic}
\end{problem}
\end{document}
