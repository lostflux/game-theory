\documentclass[11pt, reqno]{amsart}

\input{~/latex-common/macros.tex}
\usepackage[backend=bibtex,style=science]{biblatex}
% \bibliography{main.bib}
\pgfplotsset{compat=1.18}

\pagestyle{fancy}                         % fancy (allow headers, footers)
\fancyhf{}                                % clear all header/footer settings.
\cfoot{\thepage}                          % set page-numbers in footer.
% \lhead{\textit{\textbf{ Amittai, S}}}   % set name in header, left.
% \rhead{\textsc{Math 71: Algebra}}       % set class name in header, right.
\renewcommand{\headrulewidth}{0pt}
\renewcommand{\footrulewidth}{0pt}


\renewcommand{\theenumi}{\alph{enumi}}

\begin{document}

\newdate{due-date}{14}{04}{2023}

\title{CS-49: Game Theory\\ Amittai Siavava \\ \displaydate{due-date}}
\author{Amittai Siavava}
% \date{\today}


\setlength{\headheight}{13.0pt}
\setlength{\footskip}{15.0pt}

\maketitle

\bigskip

\def \cram { \textsc{cram} }
\def \dom { \textsc{domineering} }
\def \sub { \textsc{subtraction} }
\def \weighted { \textsc{weighted odds and evens}}

% \newpage
\begin{problem}[8]
  Before last basketball season, WNBA star Vera Similitude's lifetime free-throw
  percentage was below $80\%$. After the season, it was above $80\%$.
  Must there have been a moment in the season when it was exactly $80\%$?
  Justify your answer!

  \medskip
  Yes, the ratio must have been exactly $80\%$ at some point in the season.
  Suppose that Vera has $X$ free throws in total initially, of which $Y$ were successful.
  $X/Y$ is her lifelong free-throw ratio, which is below $0.8$.
  Let $n = Y - X$ be the difference between her successful and unsuccessful free throws.
  To show that the percentage must equal $80\%$:
  \begin{enumarabic}
    \item Order $n$ containers. For each free throw, fill up the containers with coins.
      But consider each container ``full'' if it has $4$ coins in it.
    \item For each new free throw that she successfully scores this season,
      add a coin to the next non-full container.
    \item For each new free throw that she misses, order one extra container
      that needs to be filled.
  \end{enumarabic}
  \medskip
  In this way, each container corresponds to a miss, and each coin corresponds to
  a successful free throw. Since each container is matched up to four coins,
  the ration is $80\%$ exactly when all the available containers have been filled
  with $4$ coins each.
  On the other hand, the ratio is \emph{above} $80\%$ only if there are coins
  that cannot be fit in a container, since all containers are already filled with
  $4$ coins each.
  But for this to happen, there must have been a moment in the season when
  all containers were filled, and no extra coin was yet to be added to a container.
  Hence, there must have been a moment in the season when the ratio was exactly $80\%$.

  \medskip

\end{problem}
\end{document}
