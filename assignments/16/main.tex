\documentclass[11pt, reqno]{amsart}

\input{~/latex-common/macros.tex}
\usepackage[backend=bibtex,style=science]{biblatex}
% \bibliography{main.bib}
\pgfplotsset{compat=1.18}

\pagestyle{fancy}                         % fancy (allow headers, footers)
\fancyhf{}                                % clear all header/footer settings.
\cfoot{\thepage}                          % set page-numbers in footer.
% \lhead{\textit{\textbf{ Amittai, S}}}   % set name in header, left.
% \rhead{\textsc{Math 71: Algebra}}       % set class name in header, right.
\renewcommand{\headrulewidth}{0pt}
\renewcommand{\footrulewidth}{0pt}


\renewcommand{\theenumi}{\alph{enumi}}

\begin{document}

\newdate{due-date}{05}{05}{2023}

\title{CS-49: Game Theory\\ Amittai Siavava \\ \displaydate{due-date}}
\author{Amittai Siavava}
% \date{\today}


\setlength{\headheight}{13.0pt}
\setlength{\footskip}{15.0pt}

\maketitle

\bigskip

\def \cram { \textsc{cram} }
\def \dom { \textsc{domineering} }
\def \sub { \textsc{subtraction} }
\def \weighted { \textsc{weighted odds and evens}}
\def \nim { \textsc{nim} }
\def \P { \mathbf{P}}
\def \N { \mathbf{N}}
\def \cc { \mathbf{cc} }

\newpage
\begin{problem}[16]

  \step
  A $\$10$ bill is auctioned to $10$ people in the following way:
  \begin{itemize}
    \item Each person chooses a non-negative integer number of dollars to submit as a sealed bid.
    \item The $\$10$ bill goes to a uniformly random person who was among the highest bidders,
    in return for the amount bid.
  \end{itemize}
  Find and count all pure Nash equilibria for this game.
  (You may assume everyone's utility for money is linear in the range $0$ to $10$.)

  \step
  If a player bids the single highest bid, then they win the bidding with probability $1$.
  However, if a player is among $x$ players with the highest bid, then the player has a probability
  $1/n$ of winning the bid.
  Therefore, whenever possible (except when they make a loss or win nothing by betting $\$10$),
  each person would prefer to bid the lowest number that is higher than all other bids.
  So, for instance, if the second-highest bid is $\$5$,
  then the winning bidder would prefer to bid $\$6$ rather than $\$9$, since in the first
  case they win $\$4$ while in the second case they only win $\$1$.
  However, in all cases, each player would prefer their bid to be \emph{at least}
  equal to the maximum bid; that way, they always have some chance of winning.

  \step
  In the Nash equilibria, no player has anything to gain by changing their bid.
  This means:
  \begin{enumroman}
    \item No player gains an advantage by raising their bid --- therefore, their bid
      must either be the maximum bid that still wins something ($\$9$) or be $\$1$
      more than all the other bids. Since the condition holds for \emph{all} players,
      it must be the first case and not the second.
    \item They do not gain an advantage by lowering their bid --- therefore, their bid
      must not be more than $\$1$ over the second-highest bid(s).
  \end{enumroman}

  With all these in mind, we can determine that the Nash equilibrium happens when
  \emph{every} player's bid is $\$9$. There is only one Nash equilibrium.
\end{problem}
\end{document}
